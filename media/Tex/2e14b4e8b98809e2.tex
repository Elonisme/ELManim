\documentclass[preview]{standalone}

\usepackage[english]{babel}
\usepackage[utf8]{inputenc}
\usepackage[T1]{fontenc}
\usepackage{lmodern}
\usepackage{amsmath}
\usepackage{amssymb}
\usepackage{dsfont}
\usepackage{setspace}
\usepackage{tipa}
\usepackage{relsize}
\usepackage{textcomp}
\usepackage{mathrsfs}
\usepackage{calligra}
\usepackage{wasysym}
\usepackage{ragged2e}
\usepackage{physics}
\usepackage{xcolor}
\usepackage{microtype}
\usepackage[UTF8]{ctex}
\linespread{1}

\begin{document}

\begin{center}
OPSIS法是用来处理指标决策问题的多方案排序和选择的方法,它的基本思想是:依据理想点的理论原理,\\找寻距离理想点最近的方案。并通过计算对象与最优解、最劣解的距离大小,确定顺序。\\即先设定一个虚拟的最优解(又称正理想解)和一个最劣解(又称负理想解),将各备选方案与正负理想解相互比较,\\若方案最靠近最优解即又距最劣解最远, 为最好。TOPSIS方法需要的评价指标决策矩阵和指标权重,\\本项目由熵权法计算给出
\end{center}

\end{document}
