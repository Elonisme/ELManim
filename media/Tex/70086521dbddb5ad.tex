\documentclass[preview]{standalone}

\usepackage[english]{babel}
\usepackage[utf8]{inputenc}
\usepackage[T1]{fontenc}
\usepackage{lmodern}
\usepackage{amsmath}
\usepackage{amssymb}
\usepackage{dsfont}
\usepackage{setspace}
\usepackage{tipa}
\usepackage{relsize}
\usepackage{textcomp}
\usepackage{mathrsfs}
\usepackage{calligra}
\usepackage{wasysym}
\usepackage{ragged2e}
\usepackage{physics}
\usepackage{xcolor}
\usepackage{microtype}
\usepackage[UTF8]{ctex}
\linespread{1}

\begin{document}

\begin{center}
\quad\\(1)The Manim Community v0.15.0 Reference Manual\quad\\(2)李丽. 基于数据挖掘的城市环境空气质量决策支持系统设计与实现[D].山东师范大学,2006.(3)马媛媛,孙世群.模糊综合评价在合肥市大气环境评价中的应用[J].环境科学与管理,2012,37(05):188-191.(4)王刚,张福印,李明辉,王金龙,王艺博,武传伟.基于偏最小二乘回归算法的空气 质 量 监 测 系 统 研 究 [J]. 传 感 器 与 微 系 统 ,2022,41(01):37-40+49.DOI:10.13873/J.1000-9787(2022)01-0037-04(5)金文彪,姚永杰,金哲植.基于主成分分析的大气环境预测研究[J].科教导刊(中旬刊),2016(32):148-150.DOI:10.16400/j.cnki.kjdkz.2016.11.071.(6)]李力争,李淑民,张晓郁,赵立娜.灰色系统在大气环境质量评价及变化趋势研究中的应用[J].环境科学与管理,2013,38(01):177-180.(7)邬红娟,林子扬,郭生练.人工神经网络方法在资源与环境预测方面的应用[J].长江流域资源与环境,2000(02):237-241.(8)袁冲.基于熵权法的江苏省各市经济高质量发展评价分析[J].商业经济,2022(04):19-20+42.DOI:10.19905/j.cnki.syjj1982.2022.04.061.(9)杨悦,杨森,杨放晴,陈鸿平,陈林,刘友平.熵权 TOPSIS 模型在竹叶花椒药材质 量 综 合 评 价 中 的 应 用 [J]. 成 都 中 医 药 大 学 学 报 ,2022,45(01):5-10.DOI:10.13593/j.cnki.51-1501/r.2022.01.005.
\end{center}

\end{document}
